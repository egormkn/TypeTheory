\documentclass[12pt,a4paper,oneside]{book}
\usepackage[utf8]{inputenc}
\usepackage[english,russian]{babel}
\usepackage[top=2cm,bottom=1cm,left=1.5cm,right=1.5cm]{geometry}

\begin{document}
\pagestyle{empty}

\begin{center}
\begin{Large}\scshape Программа курса <<Теория типов>>\end{Large}\\\vspace{0.1cm}
\textit{ИТМО, группы M3236-M3239 (year2015), весна 2017 г.}
\end{center}
\vspace{0.5cm}

\begin{enumerate}
\item Бестиповое лямбда-исчисление. Общие определения, теорема Чёрча-Россера.
\item Булевские значения, чёрчевские нумералы, упорядоченные пары, 
алгебраические типы. Нормальный и аппликативный порядок редукций, мемоизация.
\item Бета-эквивалентность и \textbf{Y}-комбинатор. Парадокс Карри.
\item Просто типизированное лямбда-исчисление. Исчисление по Чёрчу и по Карри.
Изоморфизм Карри-Ховарда. Импликационный фрагмент интуиционистского исчисления высказываний.
\item Нетипизируемость \textbf{Y}-комбинатора. Слабая и сильная нормализация. Задачи проверки типа, 
реконструкции типа, обитаемости типа в просто типизированном лямбда-исчислении
(постановка задач, общие замечания).
\item Унификация. Алгоритм нахождения типа в просто типизированном лямбда-исчислении.
\item Сильная нормализуемость просто типизированного лямбда-исчисления.
\item Логика второго порядка. Выразимость связок через импликацию и квантор всеобщности в интуиционистской логике 2-го порядка.
\item Система F. Изоморфизм Карри-Ховарда для системы F. Упорядоченные пары, алгебраические и экзистенциальные типы.
\item Типовая система Хиндли-Милнера, алгоритм W. Типизация \textbf{Y}-комбинатора.
\item Обобщённые типовые системы. Типы, рода, сорта. Зависимые типы. Лямбда-куб.
\item Язык Идрис. $\Sigma$ и $\Pi$ типы в языке Идрис. Типизация \texttt{printf} с использованием зависимых типов.
\item Доказательства в языке Идрис (на примере коммутативности сложения).
\item Теорема Диаконеску. Типы и сетоиды.
\item Линейная логика, линейные связки. Комбинаторы. Линейные и уникальные типы.
\end{enumerate}

\end{document}
